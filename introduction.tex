\chapter{Introduction}

%Workflow is becoming popular

Over the years, with the emerging of the fourth paradigm of science discovery \cite{Hey2009}, scientific workflows continue to gain their popularity among many science disciplines, including physics \cite{Deelman2002}, astronomy \cite{Sakellariou2010}, biology \cite{Lathers2006, Oinn2004}, chemistry \cite{Wieczorek2005}, earthquake science \cite{Maechling2007} and many more. Scientific workflows increasingly require tremendous amounts of data processing and workflows with up to a few million tasks are not uncommon \cite{Callaghan2011}. Among these large-scale, loosely-coupled applications, the majority of the tasks within these applications are often relatively small (from a few seconds to a few minutes in runtime). However, in aggregate they represent a significant amount of computation and data \cite{Deelman2002}. For example, the CyberShake workflow \cite{Rynge2012} is used by the Southern California Earthquake Center (SCEC) \cite{SCEC} to characterize earthquake hazards using the Probabilistic Seismic Hazard Analysis (PSHA) technique. CyberShake workflows composed of more than 800,000 jobs have been executed on the TeraGrid \cite{TeraGrid}. When executing these applications on a multi-machine parallel environment, such as the Grid or the Cloud, significant system overheads may exist. An overhead is defined as the time of performing miscellaneous work other than executing the user's computational activities.  Overheads adversely influence runtime performance of large-scale scientific workflows and cause significant resource underutilization \cite{Chen2011}. In order to minimize the impact of such overheads, task clustering \cite{Singh2008,Hussin2010,Zhao2009} and workflow partitioning \cite{Kumar2002, Hedayat2009} have been developed to increase their computational granularity and thereby reduce the system overheads. Task clustering is a technique that merges small tasks into larger jobs so that the number of computational activities is reduced. It has been widely used in executing such large-scale scientific workflows and has demonstrated its great effort \cite{Singh2008}. Workflow partitioning is the other technique that divides a large workflow into several sub-workflows such that the overall number of tasks is reduced and the resource requirements of these sub-workflows can be satisfied in the execution environments. The performance of workflow partitioning has been evaluated in \cite{Rynge2012}. 
 
However, these existing methods have a simple approach to optimize the task granularity in a job or in a sub-workflow. For example, Horizontal Clustering (HC) \cite{Singh2008} merges multiple tasks that are at the same horizontal level of the workflow. The clustering granularity (number of tasks within a cluster) of a clustered job is controlled by the user, who defines either the number of tasks per clustered job (clusters.size), or the number of clustered jobs per horizontal level of the workflow (clusters.num). However, such an empirical approach of tuning task granularity has ignored or underestimated the dynamic features of distributed environments. First of all,  such a naive setting of clustering granularity may cause significant imbalance between of runtime and data since it heavily relies on the users' knowledge. Second, these techniques have ignored the occurrence of task failures, which will counteract the benefit from task clustering. The workflow partitioning approach \cite{Rynge2012} has also ignored that the execution environments have limited resources to host some large-scale scientific workflows, in particular, the data storage constraints.


%In a DAG model, nodes represent computational activities that users want to perform. Directed edges represent control or data dependencies that must be followed between the nodes. The simplicity and  expressivity of DAG wins wide support in many workflow management systems. In this work, we focus on the extension of DAG model to be overhead aware (o-DAG). An overhead is defined as the time of performing miscellaneous work other than executing the user's computational activities. 
%In general, the execution environment of high-performance computing systems is modeled as a resource pool of multiple computational units (nodes or cpu slots namely). This is usually true when it is limited to campus-wide clusters in universities or government labs, which are typically owned, maintained, and used by a single organization. However, in recent years, two new distributed computing paradigms require a more hierarchical model of execution environments with distributed computing resources. The development of grid computing \cite{Foster2004} has enabled virtual organizations (VO) to securely share the computational infrastructure of their member institutions. More recently, the emergence of cloud computing \cite{Armbrust2009} has made it possible to easily lease large-scale computing infrastructure from commercial resource providers. However, with the increase of cloud computing. Scheduling problem is simplified as a clustering problem. 


These developments have introduced many challenges for the management of large-scale scientific workflows. In next section, we generalize these requirements to the three research challenges. 

\section{Problem Space}

In this thesis, we identify the new challenges when executing complex scientific applications:

The first challenge has to do with the \textbf{data management within a workflow}. Scientific applications are often data intensive and usually need collaborations of scientists from different institutions, hence application data in scientific workflows are usually distributed. Particularly with the emergence of grids and clouds, scientists can upload their data and launch their applications on scientific cloud workflow systems from anywhere in the world via the Internet. However, merging tasks that have input data from or have output data to different data centers is time-consuming and inefficient. Merging tasks that have no intermediate data between them seems safe at the first sight. However, the subsequent tasks that rely on the output data that their parent tasks produce may suffer a data locality problem since data may be distributed poorly and the data transfer time is increased. Communication-aware scheduling \cite{Sonmez2006, Jones2004} has taken the communication cost into the scheduling/clustering model and have achieved some significant improvement. The workflow partitioning approach \cite{Hedayat2009, Yuan2010, Wieczorek2005,Rubing2005} represents the clustering problem as a global optimization problem and aims to minimize the overall runtime of a graph. However, the complexity of solving such an optimization problem does not scale well. Heuristics \cite{Maheshwari2012, Callaghan2010} are used to select the right parameters and achieve better runtime performance but the approach is not automatic and requires much experience in distributed systems. 


The second challenge users face when merging workflow tasks is the \textbf{imbalance of runtime and data dependency}. Tasks may have diverse runtimes and such diversity may cause significant load imbalance. To address this challenge, researchers have proposed several approaches. Bag-of-Tasks \cite{Hussin2010, Celaya2010, Oprescu2010} dynamically groups tasks together based on the task characteristics but it assumes tasks are independent, which limits its usage in scientific workflows. Singh \cite{Singh2008} and Rynge \cite{Rynge2012} examine the workflow structure and groups tasks together into jobs in a static way for best effort systems. However, this work ignores the computational requirement of tasks and may end up with an imbalanced load on the resources. A popular technique in workload studies to address the load balancing challenge is over-decomposition \cite{Lifflander2012}. This method decomposes computational work into medium-grained tasks. Each task is coarse-grained enough to enable efficient execution and reduce scheduling overheads, while being fine-grained enough to expose significantly higher application-level parallelism than that is offered by the hardware. 

%However, what makes this problem even challenging is that solutions to address the data dependency challenge usually conflicts Therefore, we claim that it is necessary to consider the data dependencies with subsequent tasks (not only child tasks). However, they have forgotten balance and data structure problem. 


%\textbf{Resource management} is the third challenge that is brought by the recent emergence of cloud computing and resource provisioning techniques. Along with the increase of the scale of workflows, the number and the variety of computational resources to use has been increasing consistently. Infrastructure-as-a-Service (IaaS) clouds offer the ability to provision resources on-demand according to a pay-per-use model and adjust resource capacity according to the changing demands of the application \cite{Abrishami2012}. Task clustering can still be applied to this cloud scenario \cite{Deelman2010, Vockler2011}. However, the decisions required in cloud scenarios not only have to take into account performance-related metrics such as workflow makespan, but must also consider the resource utilization, since the resources from commercial clouds usually have monetary costs associated with them. Therefore, the adoption of task clustering on cloud computing requires the development of new methods for the integration of task clustering and resource provisioning. 

%Fault Tolerance Challenge
The third challenge has to do with \textbf{fault tolerance}. Existing clustering strategies ignore or underestimate the impact of the occurrence of failures on system behavior, despite the increasing impact of failures in large-scale distributed systems. Many researchers \cite{Zhang2004, Tang1990, Schroeder2006, Sahoo2004} have emphasized the importance of fault tolerance design and indicated that the failure rates in modern distributed systems are significant. The major concern has to do with transient failures because they are expected to be more prevalent than permanent failures \cite{Zhang2004}. For example, denser integration of semiconductor circuits and lower operating voltage levels may increase the likelihood of bit-flips when circuits are bombarded by cosmic rays and other particles \cite{Zhang2004}. In a faulty environment, there are usually three approaches for managing workflow failures. First, one can simply retry the entire job when its computation is not successful as in the Pegasus Workflow Management System \cite{Deelman2004}. However, some of the tasks within the job may have completed successfully and it could be a waste of time and resources to retry all of the tasks. Second, the application process can be periodically check-pointed so that when a failure occurs, the amount of work to be retried is limited. However, the overheads of checkpointing can limit its benefits \cite{Zhang2004}. Third, tasks can be replicated to different nodes to avoid location-specific failures \cite{Zhang2009}. However, inappropriate clustering (and replication) parameters may cause severe performance degradation if they create long-running clustered jobs. 
%Intuitively, a long-running job that consists of many tasks has a higher job failure rate even when the overall task failure rate is low. 


% After examining the major challenges in executing large-scale scientific workflows, we contribute to the studies of workflow performance improvement through task clustering in the following aspects:


\section{Thesis Statement}
This thesis states that \textbf{optimizing task granularity in scientific workflows can significantly improve the overall performance of execution}. We distinguish our work by exploring novel mechanism and new knowledge in several aspects. First, we propose data aware workflow partitioning to divide large workflows into sub-workflows that satisfy the data storage limit in the execution environments. Second, we propose a series of balanced task clustering strategies to address the tradeoff of dependency imbalance and runtime imbalance. Third, we propose fault tolerant clustering algorithms to automatically adjust the task granularity in a faulty environment and thereby improve the fault tolerance of executing scientific workflows. 

This thesis use a wide range of scientific workflows to demonstrate the efficiency and effectiveness of our approaches. 
%We believe our contributions of new task clustering and workflow partitioning mechanism and knowledge can be used by future workflow management systems to overcome resource constraints, load imbalance and improve fault tolerance. 
We believe our approaches can be used by and will inspire new ways for future optimization of task granularity in scientific workflows. 


%This thesis states that \textbf{optimization in task granularity is necessary for large-scale scientific workflows in modern distributed environments}.  First, we build an overhead and workflow model to demonstrate the reason why and how scientific workflows benefits from task clustering. Second, we develop resource aware heuristics to partition large-scale workflows into sub-workflows to satisfy resource constraints. Third, we develop a series of effective balancing algorithms to address the tradeoff of dependency imbalance and runtime imbalance. Last, we develop an effective transient failure model to study the influence of failure occurrence on task clustering to further improve the performance of task clustering in a faulty environment. This thesis use a wide range of scientific workflows to demonstrate the efficiency and effectiveness of our approaches. We believe our contributions of new task clustering mechanism and knowledge can be used by future workflow management systems to overcome resource constraints, load imbalance and improve fault tolerance. 

\section{Supporting the Thesis Statement}

In this section, we substantiate the thesis statement through four specific studies, each gaining new insights and knowledge about the task clustering in scientific workflows. 

In our first work \cite{Chen2011} (Chapter \ref{chap:model}), we \textbf{extend the existing DAG model to be overhead aware} and quantitatively analyze the relationship between the workflow performance and overheads. Previous research has established models to describe system overheads in distributed systems and has classified them into several categories \cite{Prodan2007, Prodan2008}. In contrast, we investigate the distributions and patterns of different overheads and discuss how the system environment (system configuration, etc.) influences the distribution of overheads. 
%Furthermore, we present quantitative metrics to measure and evaluate the characters (robustness, sensitivity, balance, etc.) of workflows. Finally, we analyze the relationship between these metrics and the workflow performance with different optimization methods. 

In our second work \cite{Integration2012, Chen2011a} (Chapter \ref{chap:partitioning}), we introduce \textbf{data aware workflow partitioning} to reduce the data transfer between clustered jobs. Data-intensive workflows require significant amount of storage and computation. For these workflows, we need to use multiple execution sites and consider their available storage. Data aware partitioning aims to reduce the intermediate data transfer in a workflow while satisfying the storage constraints. Heuristics and algorithms are proposed to improve the efficiency of partitioning and experiment-based evaluation is performed to validate its effectiveness.  

In our third work \cite{Chen2013a,Chen2013b} (Chapter \ref{chap:balance}), to \textbf{solve the runtime and data dependency imbalance problem}, we introduce a series of balanced clustering methods. 
%The imbalance problem means that the execution of workflows suffers from significant overheads (unavailable data, overloaded resources, or system constraints) due to inefficient task clustering and job execution. 
We identify the two challenges: runtime imbalance due to the inefficient clustering of independent tasks and dependency imbalance that is related to dependent tasks. What makes this problem even more challenging is that solutions to address these two problems are usually conflicting. For example, balancing runtime may aggravate the dependency imbalance problem, and vice versa. A quantitative measurement of workflow characteristics is required to serve as a criterion to select and balance these solutions. To achieve this goal, we propose a series of imbalance metrics to reflect the internal structure (in terms of runtime and dependency) of the workflow. 

In our fourth work \cite{Chen2012} (Chapter \ref{chap:tolerance}), we propose \textbf{fault tolerant clustering} that dynamically adjusts the clustering strategy based on the current trend of failures. During the runtime, this approach uses Maximum Likelihood Estimation to estimate the failure distribution among all the resources and dynamically merges tasks into jobs of moderate size and recluster failed jobs to avoid further failures.


Overall, this thesis aims to \textbf{improve the overall performance of task clustering and workflow partitioning in large-scale scientific workflows}. We present both experiment-based and simulation-based evaluation of a wide range of scientific workflows. 


\section{Research Contributions}

The main contribution of this thesis is a framework for task clustering and workflow partitioning in distributed autonomous systems. Specially
\begin{enumerate}
\item We have developed an overhead aware workflow model to investigate the performance of task clustering in distributed environments. We present the overhead characteristics for a wide range of widely used workflows.
% In addition, we have showed how existing workflow optimization methods improve runtime performance by reducing some or all types of overheads.
\item We have developed partitioning algorithms that use heuristics to divide large-scale workflows into sub-workflows to satisfy resource constraints such as data storage constraint. 
\item We have built a statistical model to demonstrate that transient failures can have a significant impact on the runtime performance of scientific workflows. We have developed a Maximum Likelihood Estimation based parameter estimation approach to integrate both prior knowledge and runtime feedbacks. We have proposed fault tolerant clustering algorithms to dynamically adjust the task granularity and improve the runtime performance. 
\item We have examined the reasons that cause runtime imbalance and dependency imbalance in task clustering. We have proposed quantitative metrics to evaluate the severity of the two imbalance problems and a series of balanced clustering methods to address the load balance problem for five widely used scientific workflows. 
\item We have developed an innovative workflow simulator called WorkflowSim with the implementation of popular scheduling algorithms and task clustering algorithms. 
We have built an open source community for the users and developers of WorkflowSim. 
%\item Using a set of trace-based simulations, we compare the overall performance with existing approaches for a wide range of popular scientific workflows. We show that the proposed approach can provide significant improvement for the application. 
\end{enumerate}
 
%In particular, we provide a novel approach to capture these metrics. Our work considers the neighboring tasks including siblings, parents, children and so on because such a family of tasks has strong connections between them. The performance evaluation shows that they can significantly reduce the imbalance problem and we analyze and connect the performance of these metrics and balancing methods. These quantitative metrics indicate which type of imbalance problem a workflow is more likely to suffer from. Comparing the relative values of these metrics informs the selection of a balancing method or a combination of these methods. 

%The Pegasus Workflow Management System (Pegasus WMS) is a framework for mapping complex workflows onto distributed resources such as grids and clouds. Pegasus has been used to optimize runtime performance of various scientific applications in astronomy, biology, physics, and earthquake science on dedicated clusters, and national cyberinfrastructure such as the TeraGrid and the Open Science Grid. To prepare and execute a large-scale workflows on these distributed environments has features such as the distributed nature of these resources, the large number of tasks in a workflow, and the complex dependencies among the tasks. Due to these features, significant overheads can occur during workflow execution. Failures occurring to different layers of the workflow management systems may exist. These challenges are hard to solved with conventional scheduling algorithm based approach. Instead, we aim to use a workflow restructuring approach that reorganizes workflow activities within it to improve the resource utilization, runtime performance and fault tolerance. For example, many of existing algorithms have ignored or underestimated the overheads and runtimes variabilities. 

